\documentclass[a4paper, 9pt, DIV=24]{scrartcl}
\usepackage[utf8]{inputenc}

\usepackage[ngerman]{babel}
\usepackage{amsmath}
\usepackage{amsfonts}
\usepackage{amssymb}

\usepackage{fancyhdr}

\usepackage{enumitem}

%opening
\title{Analysis II -- Zusammenfassung}
\author{Tim Taubner}

\DeclareMathOperator{\clos}{clos}
\DeclareMathOperator{\Int}{int}
\DeclareMathOperator{\arcsinh}{arcsinh}
\DeclareMathOperator{\arccosh}{arccosh}
\DeclareMathOperator{\arctanh}{arctanh}
\DeclareMathOperator{\grad}{grad}
\DeclareMathOperator{\rot}{rot}
\DeclareMathOperator{\Div}{div}

\newcommand{\N}{\mathbb{N}}
\newcommand{\R}{\mathbb{R}}
\newcommand{\C}{\mathbb{C}}

\begin{document}
\pagestyle{fancy}
\fancyhf{}
\fancyhead[L]{Zusammenfassung Analysis I, HS 2012, Analysis II, FS 2013}
\fancyhead[R]{Tim Taubner, B.\ Sc.\ ETH Informatik, Legi-Nr.: 12-937-546 -- S. \thepage}

\begin{twocolumn}
\tableofcontents
\end{twocolumn}

\clearpage

\section{Einfaches}

\subsection{Vollständige Induktion}
Kann für ein Prädikat $P(n)$ bewiesen werden, dass $P(n_0)$ und $\forall n\in\N: n > n_0 \wedge P(n) \rightarrow P(n+1)$ gilt,
dann folgt daraus $\forall n\in\N: n \geq n_0 \rightarrow P(n)$.
\paragraph{Induktionannahme (IA)} bezeichnet das Prädikat $P(n)$.
\paragraph{Induktionsverankerung (IV)} ist der Beweis von $P(n_0)$.
\paragraph{Induktionsschritt (IS)} ist der Beweis von $P(n) \rightarrow P(n+1)$.

\subsection{Logik}
\paragraph{Wahrheitstafel als Definition gängiger, bool'scher Operatoren}
\[
\begin{array}{ccccccc}
A&B&\neg A&A \wedge B&A \vee B & A \rightarrow B & A \leftrightarrow B \\\hline
0&0&1&0&0&1&1\\\hline
0&1&1&0&0&1&0\\\hline
1&0&0&0&1&0&0\\\hline
1&1&0&1&1&1&1\\
\end{array}
\]

\subsection{Mengen}
\subsubsection{Definitionen}
Seien im Folgenden $A, B$ Mengen.
\begin{enumerate}[label={(}\arabic*{)}]
 \item $A \cup B := \{x\, |\, x \in A \lor x \in B\}$ -- \emph{Vereinigung}
 \item $A \cap B := \{x\, |\, x \in A \land x \in B\}$ -- \emph{Durchschnitt}
 \item $A \backslash B := A - B := \{x\, |\, x \in A \wedge x \notin B\}$ -- \emph{Differenz}
 \item $A^C := \overline{A} := {x\, |\, x \notin A} = M \backslash A$ -- \emph{Komplement (bzgl. $M$)}
 \item $A \subseteq B := \forall x \in A: x \in B.$ -- \emph{Teilmenge}
\end{enumerate}
\subsubsection{Rechenregeln}
Diese Beweise (und ähnliche) können durch Einsetzen der obigen Definitionen und logisches Umformen geführt werden.
\begin{enumerate}[label={(}\arabic*{)}]
 \item $A \cup B = B \cup A, \\ A \cap B = B \cap A.$
 \item $A \cup (B \cup C) = (A \cup B) \cup C, \\ A \cap (B \cap C) = (A \cap B) \cap C.$
 \item $A \cup (B \cap C) = (A \cap B) \cup (A \cap C), \\ A \cap (B \cup C) = (A \cup B) \cap (A \cup C).$
 \item $(A\backslash B) \cup C = (A \cup B) \cap (B^C \cup C), \\ (A\backslash B) \cap C = A \backslash (B \cup C^C).$
 \item $(A \cup B)^C = A^C \cap B^C, \\ (A \cap B)^C = A^C \cup B^C.$
 \item $(A\backslash B) = A \cap B^C.$
 \item $(A\backslash B) \backslash C = A \backslash (B \cup C).$
\end{enumerate}
\subsubsection{Wichtige Mengen}
\begin{description}
 \item[$\N_0$, natürliche Zahlen mit 0] $\N := \{0,1,2,3,\dots\}.$
 \item[$\N$, natürliche Zahlen] $\N := \{1,2,3,\dots\} = \N\backslash\{0\}.$
 \item[$\mathbb{Z}$, ganze Zahlen] $\mathbb{Z} := \{\dots,-2,-1,0,1,2,\dots\}.$
 \item[$\mathbb{Q}$, rationale Zahlen] $\mathbb{Q} := \{\frac{p}{q}\, |\, p \in \mathbb{Z}, q \in \N_0\}.$
 \item[$\mathbb{R}$, reelle Zahlen] $\mathbb{R} := $ rationale und irrationale Zahlen, $\mathbb{Q} \subseteq \mathbb{R}$.
 \item[$\mathbb{C}$, reelle Zahlen] $\mathbb{C} := \{a - bi\, |\, a, b \in \mathbb{R}\}, $ mit $i^2 = -1$.
\end{description}
\subsubsection{Intervalle}
\begin{tabular}{ll}
$[a,b] := \{x\in\mathbb{R}\, |\, a \leq x \leq b\}$ & abgeschlossen \\
$]a,b] := \{x\in\mathbb{R}\, |\, a \leq x < b\} := (a,b]$ & halboffen (links) \\
$[a,b[\ := \{x\in\mathbb{R}\, |\, a < x \leq b\} := [a,b)$ & halboffen (rechts) \\
$]a,b[\ := \{x\in\mathbb{R}\, |\, a < x < b\} := (a,b)$ & offen \\
\end{tabular}
\begin{enumerate}[label={(}\arabic*{)}]
 \item Offene Intervalle sind offene Mengen
 \item Abgeschlossene Intervalle sind abgeschlossene Mengen
 \item Abgeschlossene, beschränkte Intervalle ($a,b \neq \infty$) sind kompakt.
\end{enumerate}
\subsubsection{Mächtigkeit}
Zwei Mengen $A, B$ heißen \emph{gleichmächtig}, wenn es eine bijektive Abbildung $f: A \rightarrow B$ gibt.
Wir schreiben $|A| = |B|$. \\
Es gilt $|\N| = |\mathbb{Z}| = |\mathbb{Q}| < |\mathbb{R}| = |[a,b]| = |\mathbb{C}|$.

\subsubsection{Topologie}
Sei im Folgenden $\Omega, A \subseteq \R^d$.
\paragraph{Definitionen}
\begin{enumerate}[label={(}\arabic*{)}]
 \item Die Menge $B_r(x_0) = \{x \in \R^d \left| |x-x_0| < r \right.\}$ heißt \emph{offener Ball} mit Radius $r > 0$ um $x_0 \in\R^d$.
 \item $x_0 \in \Omega$ heißt \emph{innerer Punkt} von $\Omega$ falls $\exists r > 0: B_r(x_0) \subseteq \Omega$.
 \item $\Omega$ heißt \emph{offen} falls alle $x \in \Omega$ innere Punkte sind.
 \item $A$ heißt \emph{abgeschlossen} falls $\R^d\backslash A$ offen ist.
 \item $\Omega^{o} := \Int(\Omega) = \bigcup_{U\subseteq\Omega,U \text{offen}} U$ heißt \emph{offener Kern} von $\Omega$.
 \item $\clos(\Omega) := \bigcap_{A\supseteq \Omega}, A \text{abgeschlossen} A$ heißt \emph{Abschluss} von $\Omega$.
 \item $\partial\Omega := \clos(\Omega)\backslash\Int(\Omega)$ heißt \emph{Rand} von $\Omega$.
 \item $\Omega$ heißt \emph{kompakt}, falls alle Folgen $(x_n) \subseteq \Omega$ ein Häufungs\-punkt (s. u.) in $K$ haben.
\end{enumerate}

\paragraph{Sätze}
\begin{enumerate}[label={(}\arabic*{)}]
 \item $\emptyset, \R^d$ sind offen und abgeschlossen.
 \item $\Omega_1, \Omega_2 \subseteq \R^d$ offen $\implies \Omega_1 \cap \Omega_2$ offen.
 \item $\Omega_i \subseteq \R^d$ offen $\implies \bigcup_{i\in I} \Omega_l$ offen.
 \item $A_1, A_2 \subseteq \R^d$ abgeschlossen $\implies A_1 \cup A_2$ abgeschlossen.
 \item $A_i \subseteq \R^d$ abgeschlossen $\implies \bigcap_{i\in I} A_l$ abgeschlossen.
\end{enumerate}


\clearpage
\section{Mittleres}

\subsection{Zwischenwertsatz}
\subsection{Folgen}
\subsubsection{Definitionen}
Falls nicht anders angegeben, ist $(a_n)_{n\in\N}$ eine Folge.
\paragraph{Grenzwert}Der \emph{Grenzwert} $a$ einer Folge existiert genau dann,
wenn 
\[ \forall \epsilon > 0 \: \exists n_0 \; \forall n > n_0: |a - a_n| < \epsilon \]
mit $ \epsilon \in \R; n,n_0 \in\N  $. \\
Wir schreiben dann $a = \lim_{n\rightarrow\infty} a_n$ oder auch $a_n \rightarrow a$.
\begin{description}
 \item[konvergent] Der Grenzwert existiert.
 \item[divergent] Der Grenzwert existiert nicht.
 \item[Nullfolge] $a = 0$.
 \item[beschränkt] $\exists C\in\R: |a_n| \leq C$.
 \item[unbeschränkt] Falls nicht beschränkt, \emph{immer} divergent!
 \item[monoton wachsend] $a_n \leq a_{n+1}, \forall{n\in\N}$
 \item[monoton fallend] $a_n \geq a_{n+1}, \forall{n\in\N}$
 \item[streng monoton wachsend] $a_n < a_{n+1}, \forall{n\in\N}$
 \item[streng monoton fallend] $a_n > a_{n+1}, \forall{n\in\N}$
 \item[alternierend] $a_n < 0 \implies a_{n+1} > 0, \forall{n\in\N}$
 \item[bestimmt divergent / uneigentlich konvergent] $a = \pm \infty$
 \item[Teilfolge] Durch Weglassen von Gliedern aus $(a_n)_{n\in\N}$ entstandene, unendliche Folge.
 \item[Häufungspunkt] $b = \lim_{n\rightarrow\infty} b_n$, $(b_n)_{n\in\N}$ Teilfolge.
 \item[$\limsup$] $\max\{b_n\text{ konvergente Teilfolge } | \lim_{n\rightarrow\infty} b_n\}$.
 \item[$\liminf$] $\min\{b_n\text{ konvergente Teilfolge } | \lim_{n\rightarrow\infty} b_n\}$.
\end{description}

\subsubsection{Konvergenzkriterien}
\begin{enumerate}[label={(}\arabic*{)}]
\item $a_n \rightarrow a \implies a_n - a \rightarrow 0 \implies |a_n - a| \rightarrow 0$.
\item Jede Teilfolge einer konvergenten Folge konvergiert gegen ihren Grenzwert.
      Eine konvergente Folge hat also genau einen Häufungspunkt.
\item $(a_n)$ monoton wachsend und nach oben beschränkt $\implies$ $(a_n)$ konvergent.
\item $(a_n)$ monoton fallend und nach unten beschränkt $\implies$ $(a_n)$ konvergent.
\item $(\sum_{n=0}^{\infty} a_n)$ konvergent $\implies$ $a = 0$, siehe Reihen.
\item $\exists f, f(n) = a_n \wedge \lim_{x\rightarrow\infty} f(x) = a \implies \lim_{n\rightarrow\infty} a_n = a$.
\item $\exists (a_n), (b_n), (c_n) \text{ mit } a_n \leq b_n \leq c_n \wedge a = c \implies b = a,$\\
      sogenanntes \textbf{Einschließungskriterium}.
\end{enumerate}

\paragraph{Cauchy-Kriterium}
Eine Folge $(a_n)_{n\in\N}$ heißt \emph{Cauchy-Folge}, falls
 \[ \forall \epsilon > 0: \exists n_0 \in\N: \forall n,l \geq n_0: |a_n - a_l| < \epsilon \]
Insbesondere gilt, $(a_n)_{n\in\N}$ konvergent $\iff (a_n)$ Cauchy-Folge.\\[.2em]
Siehe auch \textbf{Tipps an Beispielen} für angewandte Kriterien.

\subsubsection{Rechenregeln für Eigenschaften}
\paragraph{Addition}
\begin{enumerate}[label={(}\arabic*{)}]
 \item $(a_n), (b_n)$ konvergent $\implies (a_n + b_n)$ konvergent.
 \item $(a_n), (b_n)$ beschränkt $\implies (a_n + b_n)$ beschränkt.
 \item $(a_n)$ konvergent, $(b_n)$ divergent $\implies (a_n + b_n)$ divergent.
 \item $(a_n)$ beschränkt, $(b_n)$ unbeschränkt $\implies (a_n + b_n)$ unbeschränkt.
 \item $(a_n)$ beschränkt, $(b_n) \rightarrow \pm \infty \implies (a_n + b_n) \rightarrow \pm \infty$.
 \item $(a_n) \rightarrow \pm\infty$, $(b_n) \rightarrow \pm\infty \implies (a_n + b_n) \rightarrow \pm\infty$.
\end{enumerate}
\paragraph{Multiplikation}
\begin{enumerate}[label={(}\arabic*{)}]
 \item $(a_n)$ Nullfolge, $(b_n)$ beschränkt $\implies (a_n \cdot b_n)$ Nullfolge.
 \item $(a_n), (b_n)$ konvergent $\implies (a_n \cdot b_n)$ konvergent.
 \item $(a_n), (b_n)$ beschränkt $\implies (a_n \cdot b_n)$ beschränkt.
 \item $(a_n) \rightarrow a, a \neq 0$, $(b_n)$ divergent $\implies (a_n \cdot b_n)$ divergent.
\end{enumerate}
\paragraph{Grenzwerte} Wir setzen $a := \lim_{n\rightarrow\infty} a_n, b := \lim_{n\rightarrow\infty} b_n.$
\begin{enumerate}[label={(}\arabic*{)}]
 \item $\lim_{n\rightarrow\infty}(a_n \pm b_n) = a \pm b.$
 \item $\lim_{n\rightarrow\infty}(c\cdot a_n) = c\cdot a.$
 \item $\lim_{n\rightarrow\infty}(a_n \cdot b_n) = a \cdot b.$
 \item $\lim_{n\rightarrow\infty}((a_n)^c) = a^c, c$ konstant.
 \item $\lim_{n\rightarrow\infty}(\frac{a_n}{b_n}) = \frac{a}{b}, b \neq 0.$
\end{enumerate}

\subsubsection{Hilfsmethoden}
\paragraph{Referenzfolgen}
Für folgende Folgen gilt: weiter rechts stehende wachsen schneller gegen $+\infty$.
\[1, \ln(n), n^a (a > 0), q^n (q > 1), n!, n^n\]

\paragraph{Bernoullische Ungleichung}
$(1+x)^n \geq 1 + nx \quad (x \geq -1, n \in \N)$

\paragraph{Stirlingformel -- Abschätzungen für $n!$}
\[\sqrt{2\pi n}(\frac{n}{e})^n \leq n! \leq \sqrt{2\pi n}(\frac{n}{e})^n \cdot e^{\frac{1}{12n}}, \] insbesondere gilt 
$\sqrt{2\pi n}(\frac{n}{e})^n \approx n!$

\subsubsection{Tipps an Beispielen}
\paragraph{Gruppieren von Gliedern}

\paragraph{Wurzel}

\paragraph{Bruch}

\paragraph{$n$ im Exponent}

\paragraph{Satz von l'Hospital}

\subsection{Reihen}
\subsubsection{Definitionen}
Eine \emph{Reihe} $\sum_{k = 1}^\infty a_k$ heißt \emph{konvergent} mit Grenzwert $s$, wenn die Folge der \emph{Partialsummen}
$(S_n)_{n \in \N}$, $S_n := \sum_{k=1}^n a_k$ gegen $s$ konvergiert. Es gilt also wie folgt.
\[ \sum_{k=1}^\infty a_k = s \iff \lim_{n\rightarrow\infty} \sum_{k=1}^n a_k = s \]

\subsubsection{Konvergenzkriterien}

\paragraph{Nullfolge als Notwendigkeit}
Falls $(a_n)$ keine Nullfolge, gilt Folgendes nicht und somit konvergiert auch nicht folgende Reihe.
\[ \sum_{n=1}^\infty a_n \text{ konvergent} \implies \lim_{n \to \infty} a_n = 0 \]

\paragraph{$\epsilon$-Kriterium}
$\forall \epsilon > 0: \exists n_0 \in \N: \forall n \geq n_0: | \sum_{k=1}^n a_k - s | < \epsilon$

\paragraph{Absolute Konvergenz}
Konvergiert $\sum_{n=1}^\infty |a_n|$, so sagen wir die Reihe konvergiert absolut.
Es gilt $\sum_{n=1}^\infty |a_n|$ konvergent $ \implies \sum_{n=1}^\infty a_n$ konvergent.
Die Umkehrung gilt i. A. nicht.

\paragraph{Majorantenkriterium}
Ist $|a_n| \leq b_n$ und gibt es eine konvergente \emph{Majorante} $\sum_{n=1}^\infty b_n$, so konvergiert $\sum_{n=1}^\infty a_n$ absolut.

\paragraph{Minorantenkriterium}
Ist $a_n \geq b_n \geq 0$ und gibt es eine divergente \emph{Minorante} $\sum_{n=1}^\infty b_n$, so divergiert $\sum_{n=1}^\infty a_n$.

\paragraph{Leibnizkriterium}
Wenn folgende 3 Kriterien erfüllt sind, konvergiert $\sum_{n=1}^\infty a_n$.
\begin{enumerate}[label={(}\arabic*{)}]
  \item $(a_n)$ ist alternierend, also $\forall n\in\N: a_n < 0 \implies a_{n+1} > 0$
  \item $a_n \to 0$ oder $|a_n| \to 0$
  \item $(|a_n|)$ ist monoton fallend
\end{enumerate}

\paragraph{Wurzelkriterium}
\[
  \sqrt[n]{\left | a_n \right |} \to q \implies
  \begin{cases}
    q < 1 & \implies \sum_{n=1}^\infty a_n \text{ konvergiert absolut}\\
    q = 1 & \implies \text{keine Aussage}\\
    q > 1 & \implies \sum_{n=1}^\infty a_n \text{ divergiert}
  \end{cases}
\]
\paragraph{Quotientenkriterium}
\[
\left| \frac{a_{n+1}}{a_n} \right| \to q \implies
\begin{cases}
  q < 1 & \implies \sum_{n=1}^\infty a_n \text{ konvergiert absolut} \\
  q = 1 & \implies \text{keine Aussage}\\
  q > 1 & \implies \sum_{n=1}^\infty a_n \text{ divergiert}
\end{cases}
\]

\subsubsection{Potenzreihe}
Die \emph{Potenzreihe} hat die allgemeine Form \[ \sum_{n=0}^\infty a_n (x - x_0)^n = \sum_{n=0}^\infty a_n z^n, \]
dabei nennt man $x_0$ den \emph{Entwicklungspunkt}.

\paragraph{Wichtige Potenzreihen}

\begin{itemize}
 \item $ e^x = \exp(x) = \sum_{n=0}^\infty \frac{x^n}{n!} = \frac{x^0}{0!} + \frac{x^1}{1!} + \frac{x^2}{2!} + \frac{x^3}{3!} + \cdots $
\item $ \sin (x) = \sum_{n=0}^\infty (-1)^n\frac{x^{2n+1}}{(2n+1)!} = \frac{x}{1!}-\frac{x^3}{3!}+\frac{x^5}{5!}\mp\cdots $
\item $ \cos (x) = \sum_{n=0}^\infty (-1)^n\frac{x^{2n}}{(2n)!} = \frac{x^0}{0!}-\frac{x^2}{2!}+\frac{x^4}{4!}\mp\cdots $
\end{itemize}


\paragraph{Konvergenzradius} Der \emph{Konvergenzradius} sei wie folgt definiert.
\[ r:=\sup \{ |z|\ \left|\ \sum_{n=0}^{\infty}a_n z^n\ \text{ist konvergent}\right.\}\]
Es gilt also insbesondere, dass die Reihe für alle $|z| < r$ konvergiert und für für alle $|z| > r$ divergiert.
Er kann mit der \emph{Formel von Cauchy-Hadamard} wie folgt berechnet werden.
\[ r = \frac{1}{\limsup_{n\to\infty} \sqrt[n]{|a_n|}} \]
Gilt außerdem, dass ab einem $n_0\in\N$ für alle $n \geq n_0$ $a_n \neq 0$ gilt, so können wir auch wie folgt $r$ berechnen.
\[ r = \lim_{n\to\infty} \left | \frac{a_n}{a_{n+1}} \right | \]

\paragraph{Randpunkte}
Der Konvergenzradius gibt keine Hinweise auf das Konvergenzverhalten der Reihe an den sogenannten \emph{Randpunkten} $\pm r$.
Hierzu können z. B. die Randpunkte in die Reihe eingesetzt werden und anschließend die Konvergenz überprüft bzw. widerlegt werden.

\subsubsection{Rechenregeln}
Für \emph{konvergente} Reihen gilt Folgendes.
\[
  \sum_{n=1}^\infty a_n = a, \sum_{n=1}^\infty b_n = b \implies
  \sum_{n=1}^\infty (\alpha a_n + \beta b_n) = \alpha a + \beta b
\]
Für \emph{absolut konvergente} Reihen gilt außerdem, dass folgende Reihe absolut und unabhängig von der Summationsreihenfolge konvergiert.
\[
  \sum_{k,l=1}^\infty a_k b_l = \sum_{k=1}^\infty a_k \cdot \sum_{l=1}^\infty b_l
\]

\subsection{Funktionen}
Falls nicht angegeben, ist $f$ Abkürzung für $f: \Omega \mapsto \R^n$. $\Omega$ heisst dann Definitionsmenge, $\R^n$ Zielmenge.
\subsubsection{Definitionen}
\paragraph{Injektivität}
Eine Funktion heisst \emph{injektiv}, wenn jedes Element der Zielmenge \emph{höchstens} einmal als Funktionswert angenommen wird.
\paragraph{Surjektivität}
Eine Funktion heisst \emph{surjektiv}, wenn jedes Element der Zielmenge \emph{mindestens} einmal als Funktionswert angenommen wird.
\paragraph{Bijektivität}
Eine Funktion heisst \emph{bijektiv}, wenn jedes Element der Zielmenge \emph{genau einmal} als Funktionswert angenommen wird. Man kann zu einer Bijektion immer eine Umkehrfunktion finden.
\subsubsection{Grenzwerte}

\subsubsection{Stetigkeit}
\paragraph{$\epsilon$-$\delta$-Kriterium} $\lim_{x\rightarrow x_0} f(x) = a, $ wenn Folgendes gilt.
\[\forall\epsilon>0:\exists\delta>0: \forall x \in \Omega: |x-x_0| < \delta \implies |f(x) - f(x_0)| < \epsilon \]
\paragraph{Definition}
\begin{enumerate}[label={(}\arabic*{)}]
 \item $\lim_{x\rightarrow x_0} f(x) = f(x_0) \implies f(x)$ \emph{stetig} im Punkt $x_0$.
 \item $\lim_{x\rightarrow x_0} f(x) = a \implies f(x)$ \emph{stetig ergänzbar} in $x_0$.
 \item $\forall x_0\in\Omega: f(x) $ stetig in $x_0 \implies f(x)$ \emph{stetig}.
\end{enumerate}

\paragraph{Sätze über punktweise Stetigkeit}
Sei $f$ stetig in $x_0$.
\begin{enumerate}[label={(}\arabic*{)}]
 \item $\lim_{x\rightarrow x_0} f(x) = f(\lim_{x\rightarrow x_0} x)$.
 \item $\lim_{x\nearrow x_0} f(x) = f(x_0) = f(\lim_{x\searrow x_0} x)$ wenn existent.
 \item $\lim_{n\rightarrow \infty} x_n = x_0 \implies \lim_{n\rightarrow \infty} f(x_n) = f(x_0)$, für alle Folgen $(x_n)$. -- \emph{Folgenkriterium}.
\end{enumerate}

\paragraph{Gleichmäßige Stetigkeit}
$f$ heißt \emph{gleichmäßig stetig}, wenn gilt:
\[\forall\epsilon>0\exists\delta>0: \forall x,y \in \Omega: |x-y| < \delta \implies |f(x) - f(y)| < \epsilon \]
Unterschied zur punktweisen Stetigkeit ist, dass $\delta$ unabhängig von der Wahl von $y$ bzw. $x_0$ ist.

\paragraph{Lipschitz-Stetigkeit}
$f: \Omega \subseteq \R^d \rightarrow \R^n$ heißt \emph{Lipschitz-stetig} mit \emph{Lipschitz-Konstante} $L$, wenn gilt:
\[ \forall x,y\in\Omega: \|f(x)-f(y)\| \leq L\|x-y\|\]

\paragraph{Lokale Lipschitz-Stetigkeit}
$f: \Omega \subseteq \R^d \rightarrow \R^n$ heißt \emph{lokal Lipschitz-stetig},
falls zu jedem $x_0\in\Omega$ eine Umgebung $U = B_r(x_0)\cap\Omega$ existiert,
so dass $f|_U : x \in U \mapsto f(x) \in \R^n$ Lipschitz-stetig ist.


\paragraph{Sätze über gleichmäßige und Lipschitz-Stetigkeit}
\begin{enumerate}[label={(}\arabic*{)}]
 \item Ist $f$ Lipschitz-stetig mit Konstante $L$, so ist $f$ gleichmäßig stetig, z. B. mit $\delta = \epsilon/L$.
 \item Ist $f$ gleichmäßig stetig, dann ist $f$ in $\Omega^C$ stetig ergänzbar.
 \item Ist umgekehrt $\Omega$ beschränkt, $f$ stetig und in $\Omega^C$ stetig ergänzbar, so ist $f$ auch gleichmäßig stetig.
\end{enumerate}

\subsubsection{Folgen von Funktionen}
Eine \emph{Funktionsfolge} $(f_n)_{n\in\N}$ \emph{konvergiert punktweise} auf $I \subseteq \R$ gegen $f$, wenn $\forall x\in I: f_n(x) \rightarrow f(x)$:
\[ \forall \epsilon > 0: \forall x \in I: \exists n_0 \in \N: n \geq n_0 \implies |f_n(x)-f(x)| < \epsilon \]
Sie \emph{konvergiert gleichmäßig} auf $I$ gegen $f$, wenn $sup_{x\in I} |f_n(x) - f(x)| \rightarrow 0$ gilt.
Insbesondere ist also das $n_0$ nicht mehr abhängig von einem $x$.
\[ \forall \epsilon > 0: \exists n_0 \in \N: \forall x \in I: n \geq n_0 \implies |f_n(x)-f(x)| < \epsilon \]

\emph{Tipp: } Gleichmäßige Konvergenz kann häufig durch Setzen von $x := n$, oder $x := \frac{1}{n}$ widerlegt werden.
Denn $|f_n(x) - f(x)|$ muss gegen Null streben, was dann aber nicht der Fall ist.

\subsubsection{Differentialrechnung}

\paragraph{Definition}
Wir sagen $f: I \rightarrow \R$ ist in $x_0 \in I$ \emph{differenzierbar}, wenn folgender Grenzwert existiert.
\[
 \lim_{x\rightarrow x_0} \dfrac{f(x) - f(x_0)}{x - x_0} =: \dfrac{f}{fx} f(x_0) =: f'(x_0)
\]
Ist $f$ für alle $x_0 \in I$ differenzierbar, heißt die Funktion selbst \emph{differenzierbar}.
Dann ist die Funktion $f'(x)$ die \emph{Ableitung} von $f$.
Gilt außerdem, dass $f'(x)$ stetig ist, so ist $f$ \emph{stetig differenzierbar}.

\paragraph{Mittelwertsatz -- Satz von Lagrange}
Ist $f$ auf $[a,b]$ stetig und in $]a,b[$ differenzierbar, so gibt es ein $c \in ]a,b[$ mit
\[ \dfrac{f(b) - f(a)}{b-a} = f'(c) \]
Anders gesagt gibt es ein $c$, an dem die Steigung gerade die mittlere Steigung beträgt.\\
\emph{Bemerkung:} Der Mittelwertsatz kommt häufig bei Ungleichungen zur Anwendung.

\paragraph{Monotonie}
Das Monotonie-Verhaltens lässt sich anhand der 1. Ableitung bestimmen.
\begin{enumerate}[label={(}\arabic*{)}]
 \item $f' > 0 \implies $ f streng monoton steigend.
 \item $f' < 0 \implies $ f streng monoton fallend.
 \item $f' \geq 0 \iff $ f monoton steigend.
 \item $f' \leq 0 \iff $ f monoton fallend.
\end{enumerate}

\paragraph{Konvexität}
Die Konvexität lässt sich anhand der 2. Ableitung bestimmen.
Dabei heißt eine Funktion $f$ \emph{konvex}, wenn $\forall a,b: f(\frac{a+b}{2}) \leq \frac{f(a)+f(b)}{2}$
und \emph{konkav}, wenn $\forall a,b: f(\frac{a+b}{2}) \geq \frac{f(a)+f(b)}{2}$.
Insbesondere ist der Graph einer konvexen Funktion \emph{linksgekrümmt} und der einer konkaven \emph{rechtsgekrümmt}.
\begin{enumerate}[label={(}\arabic*{)}]
 \item $f'' \geq 0 \iff $ f konvex.
 \item $f'' \leq 0 \iff $ f konkav.
\end{enumerate}

\paragraph{Extremstellen}
Für Extremstellen -- also Sattelpunkte, Minima und Maxima -- von $f$ gilt $f'(x_0) = 0$.
Weitere Eigenschaften sind folgend zusammengefasst.
\begin{enumerate}[label={(}\arabic*{)}]
 \item $f''(x_0) > 0 \implies$ \emph{Minimum} bei $x_0$.
 \item $f''(x_0) < 0 \implies$ \emph{Maximum} bei $x_0$.
 \item $f''(x_0) = 0 \vee f'''(x_0) \neq 0 \implies$ \emph{Sattelpunkt} bei $x_0$.
\end{enumerate}

Aus $(3)$ folgt ohne der Voraussetzung von $f'(x_0) = 0$ übrigens, dass bei $x_0$ ein \emph{Wendepunkt} vorliegt,
also die Funktion von konvex nach konkav bzw. anders herum wechselt.

\subsection{Taylorreihe \& -entwicklung}
Funktionen lassen sich in der Umgebung eines Punktes durch eine Potenzreihe annähern.

Die \emph{Taylorreihe} von $f$ um den Punkt $x_0$ ist definiert durch:
\[ T f(x) = \sum_{n=0}^{\infty} \dfrac{f^{(n)}\cdot a}{n!}(x-x_0)^n \]
Insbesondere nennen wir die \emph{Linearisierung der Taylorreihe} mit Grad $m$ das \emph{$m$-te Taylorpolynom}.
Es ist also:
\[ T_m f(x) = \sum_{i=0}^{m} \dfrac{f^{(n)}\cdot a}{n!}(x-x_0)^n = f(x_0) + \dfrac{f'(x_0)}{1} + \dfrac{f''(x_0)}{2} + \cdots\]

\paragraph{Restglied}

\[ R_{m}(x) = \frac{f^{(m+1)}(x)}{(m+1)!}(x-a)^{m+1} \]

\paragraph{Rechenregeln}
\begin{enumerate}[label={(}\arabic*{)}]
 \item $T_m(f+g)(x) = T_m f(x) + T_m g(x)$ -- \emph{Addition}
 \item $T_m(f\cdot g)(x) = T_m f(x) \cdot T_m g(x)$, entferne alle Terme der Ordnung $> m$ -- \emph{Multiplikation}
 \item Im Allgemeinen gilt $f(x) = T f(x)$ nicht. Außerdem kann der Konvergenzradius $0$ betragen.
\end{enumerate}

\clearpage
\section{Schweres}

\subsection{Integration}
Im Folgenden seien $F, f$ definiert auf $]a,b[$.
\begin{enumerate}[label={(}\arabic*{)}]
 \item $F$ heißt Stammfunktion von $f$ falls $F' = f$.
 \item Für Stammfunktionen $F_1, F_2$ von f gilt: $F_1 - F_2$ konstant.
 \item $\int_{x_0}^{x_1}f(x)dx := F(x_1) - F(x_0)$ heißt Integral von $f$ über $[x_0, x_1]$.
   Dabei ist $a < x_0 <= x_1 < b$ und $F' = f$.
 \item \textbf{Hauptsatz:} $F(y) = \int_{a}^{y}f(x)dx, y \in\ ]a,b[ \implies F' = f.$
\end{enumerate}

\subsubsection{Rechenregeln}
Das Integral ist ein \emph{lineares} und \emph{monotones} Funktional, wie folgende zwei Sätze zeigen!
\paragraph{Linearität}
$\int_{x_0}^{x_1}\alpha f(x) + \beta g(x) dx = \alpha\int_{x_0}^{x_1}f(x)dx + \beta\int_{x_0}^{x_1}g(x)dx.$
\paragraph{Monotonie}
Sei $f,g : ]a,b[ \mapsto \mathbb{R}$ beschränkt und R-integrabel dann gilt
$ f \leq g \implies \int_{x_0}^{x_1}f(x)dx \leq \int_{x_0}^{x_1}g(x)dx. $
\paragraph{Gebietsadditivität}
$\int_{x_0}^{x_2}f(x)dx = \int_{x_0}^{x_1}f(x)dx + \int_{x_1}^{x_2}f(x)dx$, wobei $x_0 \leq x_1 \leq x_2$.

\paragraph{Substitution}
Ausgehend von der Ableitungsregel $f'(g(x)) = f'(g(x))g'(x)$ können wir folgende Integrationsregel herleiten.
\[ \int_{x_0}^{x_1}f'(g(x))g'(x)dx = f(g(x))|_{x_0}^{x_1} = \int_{g(x_0)}^{g(x_1)}f'(u)du \]
Substituiert man $u := g(x)$, ergibt sich $\frac{du}{dx} = g'(x) \iff du = g'(x)dx$.
Bleibt noch ein Restterm $i(x)$, löse $u = g(x)$ nach $x = h(u)$ auf und ersetzte $i(x)$ durch $h(i(x))$.

Die neuen Grenzen -- nur bei bestimmten Integralen -- sind nun $g(x_0)$ und $g(x_1)$.
Bei unbestimmten Integralen müssen keine Grenzen angepasst werden!

Nach Berechnung des Integrals resubstituiere $u$ durch $g(x)$.

\paragraph{Partielle Integration}
So ähnlich lässt sich auch aus der Ableitungsregel $\frac{d}{dx}f(x)g(x) = f'(x)g(x) + f(x)g'(x)$ eine Integrationsregel aufstellen.
\[ \int_{x_0}^{x_1}(f'(x)g(x)+f(x)g'(x))dx = f(x)g(x)|_{x_0}^{x_1} \]
\[ = \int_{x_0}^{x_1}f'(x)g(x)dx + \int_{x_0}^{x_1}f(x)g'(x)dx \]
\[ \iff \int_{x_0}^{x_1}f'(x)g(x)dx = f(x)g(x)|_{x_0}^{x_1} - \int_{x_0}^{x_1}f(x)g'(x)dx. \]

\subsection{Differentialgleichungen}

\subsubsection{DGL erster Ordnung}
\paragraph{Definition}
Eine Gleichung, in der (ausschließlich) die Unbekannten $y = y(x), y' = y'(x)$ und $x$ vorkommen, heißen \emph{Differentialgleichung erster Ordnung}.
\paragraph{Seperation der Variablen}
$y' = g(y)f(x)$ lässt sich mittels \emph{Seperation der Variablen} lösen.
Dazu bringen wir die ``ys auf die eine, die xs auf die andere Seite'' der Gleichung.
Anschließend integrieren wir auf beiden Seiten nach $dx$ und erhalten so Folgendes.
\[ y' = g(y)f(x) \iff \frac{y'}{g(y)} = f(x) \iff \int\frac{y'}{g(y)}dx = \int f(x)dx \]
\[ \iff \int\frac{1}{g(y)}dy = F(x) + C_0 \iff ln |g(y)| = F(x) + C_1. \]
Durch Anwenden von $exp$ auf beiden Seiten und anschließendes Umformen der Konstanten, erhalten wir schließlich.
\[ g(y) = C\cdot e^{F(x)} \iff y = g^{-1}(C\cdot e^{F(x)}) \]
\emph{Bemerke,} dass es zusätzliche, konstante Lösungen für $y$ geben kann, nämlich für alle $y$ mit $g(y) = 0$.

\paragraph{Variation der Konstanten}
Für $y' = y + x$ betrachte die Lösung der linearen, homogenen DGL $y' - y = 0$. Diese hat ungefähr die Form $y = C_1 e^{\lambda_1 x} + C_2 e^{\lambda_2 x}$.
Nun ersetze $C_1 := u_1(x), C_2 := u_2(x)$ und löse anschließend das Gleichungssystem.
\[
\begin{pmatrix}
 b \\ c \\
\end{pmatrix}
\]

\subsubsection{Lineare, homogene DGL beliebiger Ordnung}
\paragraph{Definition} Eine lineare, homogene DGL der Ordnung $n$ über eine Funktion $f \in C^{n}$ ist eine Gleichung der Form
\[ a_n f^{(n)}(x) + a_{n-1} f^{(n-1)}(x) + \dots + a_1 f'(x) + a_0 f(x) = 0. \]
\paragraph{Lösungsansatz}
Der Lösungsansatz für homogene DGL basiert auf einer Eigenwertberechnung über das charakteristische Polynom.
Man berechne die Eigenwerte $\lambda_1, \dots, \lambda_l$ mit Vielfachheiten $c_1, \dots, c_l$ durch Lösen von $a_n\lambda^n + \dots + a_0\lambda^0 = 0$.
Es gilt jetzt:
\[ f(x) = \sum_{i=1}^{l}\sum_{j=1}^{c_l}k_{i,j}x^{j-1}e^{\lambda_l x} \]
\[ = k_{1,0}e^{\lambda_1 x} + k_{1,1}xe^{\lambda_1 x} + \dots + k_{1,c_1-1}x^{c_1-1}e^{\lambda_l x} + \cdots \]
\paragraph{Partikuläre Lösung für Anfangswertproblem}
Haben wir auch $f(0) = w_0, f'(0) = w_1, \dots, f^{(n)}(0) = w_n$ gegeben,
können wir die Koeffizienten $k_{i,j}$ wie folgt ausrechnen.
Durch Lösen des folgenden Gleichungssystems erhalten wir dann die entsprechenden Koeffizienten.
\[
 \begin{pmatrix}
  f(0) \\
  f'(0) \\
  \vdots \\
  f^{(n)}(0) \\
 \end{pmatrix}
  =
 \begin{pmatrix}
  w_0 \\
  w_1 \\
  \vdots \\
  w_n \\
 \end{pmatrix}
\]

\clearpage
\subsection{Differentialrechnung in $\mathbb{R}^n$}
\subsubsection{Definitionen}
Sei im Folgenden $\Omega \subseteq \R^n, F: \Omega \mapsto \R^m$ und $f: \Omega \mapsto \R$. \\
Betrachte $f$ als $f(x_1,x_2\dots,x_n)$, dann heißt $f$ \emph{partiell differenzierbar} in Richtung $(0,\dots,e_i,\dots,0)$ bzw. nach $x_i$,
wenn die Funktion $g: x \mapsto f(x_1,\dots,x_{i-1},x,x_{i+1},\dots,x_n)$ differenzierbar ist. Wir betrachten dabei $x_0,\dots,x_{i-1},x_{i+1},\dots,x_n$ als Konstanten.

\begin{enumerate}[label={(}\arabic*{)}]
 \item $F$ wie oben heißt \emph{Vektorfunktion}.
 \item Bei $m = 1$ sprechen wir von einem \emph{Skalarfeld}.
 \item Bei $n = m$ sprechen wir von einem \emph{Vektorfeld}.
 \item Es gilt $\dfrac{\partial^2 f}{\partial x\partial y} = \dfrac{\partial^2 f}{\partial y\partial x}$. -- \emph{Satz von Schwarz}
\end{enumerate}

\subsubsection{$\nabla$-Operator, Gradient, Divergenz, Rotation}
\begin{description}
 \item[Nabla-Operator] $\nabla := (\frac{\partial}{\partial x_1},\dots,\frac{\partial}{\partial x_n})$, nur im Kartesischem!
 \item[Gradient(enfeld)] $\grad(f) := \nabla f = \begin{pmatrix}\frac{\partial f}{\partial x_1} \\\vdots\\ \frac{\partial f}{\partial x_n}\end{pmatrix}$
 \item[Divergenz] $\Div(F) := \sum_{i=1}^n\frac{\partial F_i}{\partial x_i},\, \Div(F) = \langle \nabla, F \rangle$
 \item[Rotation] $\rot(F) := \begin{pmatrix}
\frac{\partial F_3}{\partial x_2} - \frac{\partial F_2}{\partial x_3} \\
\frac{\partial F_1}{\partial x_3} - \frac{\partial F_3}{\partial x_1} \\
\frac{\partial F_2}{\partial x_1} - \frac{\partial F_1}{\partial x_2} \\
\end{pmatrix},\, \rot(F) = \nabla \times F, n = 3$
\end{description}

\subsubsection{Gradienten- /Potentialfeld und konservative Vektorfelder}
Ist $v = \grad(f)$, heißt $f$ das \emph{Potential} oder die \emph{Stammfunktion} zu dem \emph{Gradientenfeld} bzw. dem \emph{Potentialfeld} $v$.
\begin{enumerate}[label={(}\arabic*{)}]
 \item $v$ heißt \emph{konservatives Vektorfeld}.
 \item $v$ ist \emph{wirbelfrei}: $\rot(\grad f) = \vec{0}$. -- \emph{hinreichendes Kriterium}
 \item Kurvenintegrale nur abhängig von Anfangs- und Endpunkt.
 \item Kurvenintegrale mit Anfangspunkt $=$ Endpunkt sind $0$.
\end{enumerate}

\subsubsection{Jacobi-Matrix}
Die Ableitungsmatrix einer differenzierbaren Funktion $f: \R^n \rightarrow \R^m$ ist die $m\times n$-Matrix der einfachen partiellen Ableitungen.
\[
 J_f(a) := (\dfrac{\partial f_i}{\partial x_j})_{i = 1,\dots,m, j = 1,\dots,n} =
 \begin{pmatrix}
  \dfrac{\partial f_1}{\partial x_1} & \dfrac{\partial f_1}{\partial x_2} & \dots & \dfrac{\partial f_1}{\partial x_n} \\
  \vdots & \vdots & \ddots & \vdots \\
  \dfrac{\partial f_m}{\partial x_1} & \dfrac{\partial f_m}{\partial x_2} & \dots & \dfrac{\partial f_m}{\partial x_n} \\
 \end{pmatrix}\]

\subsubsection{Hesse-Matrix}
Die Hesse-Matrix ist das Analogn im $\R^n$ zur zweiten Ableitung einer eindimensionalen Funktion.
Ist $f(x_1,\dots,x_n), f: D \subseteq \R^n \rightarrow \R$ zweimal stetig diff'bar, definieren wir die quadratische Matrix $H_f$ wie folgt.
\[ H_f := (\dfrac{\partial^2 f}{\partial x_i\partial x_j})_{i,j=1,\dots,n} = \begin{pmatrix}
 \dfrac{\partial^2 f}{\partial x_1\partial x_1} & \dots & \dfrac{\partial^2 f}{\partial x_1\partial x_n} \\
 \vdots & \ddots & \vdots \\
 \dfrac{\partial^2 f}{\partial x_n\partial x_1} & \dots & \dfrac{\partial^2 f}{\partial x_n\partial x_n} \\
\end{pmatrix}
\]
Wegen des Satzes von Schwarz ist $H_f$ auch symmetrisch.
Insbesondere ist für eine Funktion $f(x,y)$
\[ H_f = \begin{pmatrix}
\dfrac{\partial^2 f}{\partial^2 x} & \dfrac{\partial^2 f}{\partial x\partial y} \\
\dfrac{\partial^2 f}{\partial x\partial y} & \dfrac{\partial^2 f}{\partial^2 y} \\
\end{pmatrix}. \]

\subsubsection{Kritische Punkte}
Im Prinzip wie im eindimensionalem, wir bestimmen Minima und Maxima durch finden der Nullstellen der Ableitung.
Allerdings müssen wir den Rand natürlich speziell betrachten, insbesondere für das Finden globaler Extrema.

\paragraph{Vorgehen}
\begin{enumerate}[label={(}\arabic*{)}]
 \item $\grad(f) = \nabla f = \vec{0}$ ergibt die Menge kritischen Punkte $K$.
 \item Hesse-Matrix $H_f$ berechnen.
 \item $det(H_f)$ berechnen.
 \item Für jedes $k\in K$ in $det(H_f)$ einsetzen:
 \item Gilt $det(H_f) < 0: k$ ist Sattelpunkt von $f$.
 \item Gilt $det(H_f) > 0 \wedge Spur(H_f) > 0: k$ ist Minimum von $f$.
 \item Gilt $det(H_f) > 0 \wedge Spur(H_f) < 0: k$ ist Maximum von $f$.
 \item Gilt $det(H_f) = 0$, keine Aussage, weiteres Vorgehen nötig.
\end{enumerate}

\subsubsection{Globale Extrema}
Angenommen, wir haben bereits alle lokale Extrema berechnet (wie oben).

\subsubsection{Tangentialebene}
Zusätzlich zu den kritischen Punkten, kann gefordert sein, die sogenannte \emph{Tangentialebene} durch den Punkt $(x_1, x_2, \dots, x_n)$ zu bestimmen.
Hier ist das Verfahren im $\R^2$ angegeben.
Sei $f(x,y) = z$ und der Punkt $(x_0, y_0, f(x_0, y_0)) = (x_0, y_0, z_0)$ gegeben.
\begin{enumerate}[label={(}\arabic*{)}]
 \item Bestimme $\grad(f) = (\frac{\partial f}{\partial x}, \frac{\partial f}{\partial y})^T := (f_x, f_y)^T$.
 \item Bilde $z(x_0, y_0), z_x(x_0, y_0)$ und $z_y(x_0, y_0)$.
 \item Stelle die \emph{Tangentialgleichung} $z = f(x_0, y_0) + f_x(x_0, y_0)(x - x_0) + f_y(x_0, y_0)(y - y_0)$ auf.
\end{enumerate}
Setzt man $F(x,y,z) = f(x,y) - z = 0$, lässt sich die \emph{Tangentialgleichung} auch wie folgt (in Normalform) darstellen.
\[ \langle (r - (x_0, y_0, z_0)), \grad(F) \rangle = 0, \grad(F) = \begin{pmatrix}f_x \\ f_y \\ -1\end{pmatrix} \]

\subsection{Potentialbestimmung}
Im Prinzip ist das Bestimmen eines Potential auch eine Art Integration.

\paragraph{Bestimmen eines Potentials im $\R^2$}
Ist ein dreidimensionales Vektorfeld $F(x,y): M \subseteq \R^2 \rightarrow \R^3$ gegeben und es soll bestimmt werden ob
es ein -- und wenn ja, welches -- Potential $f$ besitzt so dass $F = grad f$.
\begin{enumerate}[label={(}\arabic*{)}]
 \item Prüfe ob $\dfrac{\partial F_y}{\partial x} - \dfrac{\partial F_x}{\partial y} = 0$ ergibt.
 \item Ist dies nicht der Fall, so gibt es kein Potential.
 \item Sonst $f_1 = \int F_xdx + c_1$ und $f_2 = \int F_ydy$.
 \item Gleichsetzen von $f_1 = f_2$ ergibt die Konstanten $c_1, c_2$.
\end{enumerate}

\paragraph{Bestimmen eines Potentials im $\R^3$}
Ist ein dreidimensionales Vektorfeld $F(x,y,z): M \subseteq \R^3 \rightarrow \R^3$ gegeben und es soll bestimmt werden ob
es ein -- und wenn ja, welches -- Potential $f$ besitzt so dass $F = grad f$.
\begin{enumerate}[label={(}\arabic*{)}]
 \item Ist $F$ wirbelfrei? Also zeige, dass $rot F = 0$.
 \item Falls $rot F \neq 0$ sind wir fertig, denn es gibt kein Potential.
 \item Sonst $f_1 = \int F_xdx + c_1$, $f_2 = \int F_ydy + c_2$, $f_3 = \int F_zdz + c_3$.
 \item Setze nun $f_1 = f_2 = f_3$ gleich und berechne die Integrationskonstanten $c_1, c_2, c_3$.
\end{enumerate}
\emph{Bemerkung:} $M$ muss einfach zusammenhängend sein, was bei $M = \R^3$ gegeben ist.

\subsection{Kurvenintegrale}
\subsubsection{1. Art -- Wegintegral über Skalarfeld}
Das \emph{Wegintegral 1. Art} über ein stetiges Skalarfeld $f: \R^n\rightarrow\R$ entlang des stetig differenzierbaren Weges $\gamma: [a,b]\rightarrow\R^n$
ist definiert durch
\[ \int_\gamma f(s)ds := \int_a^b f(\gamma(t))\|\gamma'(t)\|_2dt. \]
Dabei ist $\|a\|_2 = \sqrt{\sum_{i=1}^n a_i^2}$ die Euklidische Norm.

\subsubsection{2. Art -- Wegintegral über Vektorfeld}
Das \emph{Wegintegral 2. Art} über ein stetiges Vektorfeld $F: \R^n\rightarrow\R^n$ entlang eines stetig differenzierbaren Weges $\gamma: [a,b]\rightarrow\R^n$
ist definiert durch
\[ \int_\gamma F(s)ds := \int_a^b\langle F(\gamma(t)), \gamma'(t) \rangle dt\]
Dabei ist $\langle a,b \rangle = a^Tb = \sum_{i=0}{n} = a_ib_i$ das (euklidische) Skalarprodukt.

\subsubsection{Rechenregeln}
Kurvenintegrale sind genauso wie ``normale'' Integrale linear.
\begin{enumerate}[label={(}\arabic*{)}]
 \item $\int_\gamma F(s) + G(s)ds = \int_\gamma F(s)ds + \int_\gamma G(s)ds$.
 \item $\int_\gamma \alpha F(s)ds = \alpha\int_\gamma F(s)ds$
\end{enumerate}

\subsection{Volumen- und Flächenintegrale im $\R^n$}

\subsubsection{Koordinatentransformation}
\paragraph{Diffeomorphismus}
$\Phi: \Omega \mapsto \Phi(\Omega) \subseteq \R^n$ heißt Diffeomorphismus, wenn $\Phi$ bijektiv und $\Phi^{-1}$ differenzierbar ist.

\paragraph{Transformationssatz}
$f: \Phi(\Omega) \mapsto \R^n$ ist genau dann integrierbar, wenn $g(x) = f(\Phi(x))|\det(J_\Phi(x)|$ integrierbar ist.
Es gilt:
\[ \int_{\Phi(\Omega)} f(x)dx = \int_{\Omega} f(\Phi(x))|\det(J_\Phi(x))|dx\]
Dies nutzen wir aus, um Integrale durch geeignete Wahl von $\Phi$ zu vereinfachen.
Dabei ist $J_\Phi(x)$ die Jacobi-Matrix von $\Phi$, siehe oben.
Für Kugel- und Zylinderkoordinaten, siehe Anhang \emph{Formeln und Tafeln}.

\subsubsection{Satz von Gauß}
Sei $\Omega \subseteq \R^n$ eine kompakte Menge mit ``glattem'' Rand $S := \partial\Omega$.
Sei weiter $F: \Omega \rightarrow \R^n$ ein stetig differenzierbares Vektorfeld. Es gilt
\[ \int_\Omega \Div FdV = \int_S F\cdot N dS \]
Wobei $N$ die Normale (an der jeweiligen Stelle) ist.

\subsubsection{Massenmittelpunkt im $\R^n$}
Sei der Vektor $r$, der \emph{Massenmittelpunkt} oder auch \emph{Schwerpunkt} eines Körpers $K$ mit \emph{Dichtefunktion} $\rho(v)$.
Dann gilt für seine Komponenten:
\[ r_x = 1/M\cdot\int_K x\cdot \rho(v)dV \]
\[ r_y = 1/M\cdot\int_K y\cdot \rho(v)dV \]
\[ r_z = 1/M\cdot\int_K z\cdot \rho(v)dV \]

Wobei die Masse $M$ gegeben ist durch:
\[ M = \int_K \rho(v) dV \]

\emph{Bemerkung} Bei homogen-dichten Körpern, also Körper, bei dem überall die gleiche Dichte gegeben ist,
lässt sich $\rho$ vor das Integral ziehen.

\clearpage
\section{Formeln und Tafeln}
Hier ist alles nur aufgelistet, für Begründungen an der jeweiligen Stelle nachgucken!
\subsection{Rechentricks}
\subsubsection{Fakultät, Binomialkoeffizienten}
\[ n! = n\cdot(n-1)\cdot\dots\cdot1, n\in\N \]
\[ \binom{n}{k} = \frac{n!}{k!(n-k!)} = \binom{n}{n-k},\ 0 \leq k \leq n \]

\subsubsection{Mitternachtsformel}
\[ ax^2 + bx + c = 0 \iff x_{1,2} = \frac{-b \pm \sqrt{b^2 - 4ac}}{2a} \]

\subsubsection{Partialbruchzerlegung}
\paragraph{Sonderfall Nenner Grad zwei}
\[
 B = \dfrac{a_zx+b_z}{(a_1x+b_1)(a_2x+b_2)} = \dfrac{u}{(a_1x+b_1)} + \dfrac{v}{(a_2x+b_2)},
\]
mit $ua_1 + va_2 = a_z \wedge ub_1 + vb_2 = b_z$.
\paragraph{Allgemeiner Fall}
Betrachte den Bruch $\dfrac{z(x)}{n(x)}$, wobei $z,n$ Polynome mit Grad $n,m$ sind.
\subparagraph{Fall 1: $n \geq m$}
Dividiere $\dfrac{z(x)}{n(x)} = v(x) + \dfrac{u(x)}{n(x)}$.
Ist $u(x) \neq 0$, so fahre mit $\dfrac{u(x)}{n(x)}$ wie in \emph{Fall 2} weiter, sonst sind wir fertig.
\subparagraph{Fall 2: $n < m$}
Faktorisiere $n(x)$ in seine $i$ Nullstellen: $n(x) = (x-x_1)^{r_1} \cdot (x-x_2)^{r_2} \cdot \dots \cdot (x-x_i)^{r_i}.$
Jetzt lösen wir das folgende Gleichungssystem durch Ausmultiplikation.
\[ \frac{a_1}{(x-x_1)^{r_1}} + \frac{a_2}{(x-x_2)^{r_2}} + \dots + \frac{a_i}{(x-x_i)^{r_i}} = \frac{z(x)}{n(x)} \]

\subsubsection{Ungleichungen}
\begin{enumerate}[label={(}\arabic*{)}]
 \item $a < b \iff a + c < b + c$, genauso für $\leq,=,>,\geq$
 \item $a < b \wedge c > 0 \iff \frac{a}{c} < \frac{b}{c}$
 \item $a < b \wedge c < 0 \iff \frac{a}{c} > \frac{b}{c}$
 \item $|a+b| \leq |a| + |b|$ -- \emph{Dreiecksungleichung}
 \item $|x\cdot y| \leq \|x\|\cdot\|y\|, x,y \in \mathbb{R}^n$ -- \emph{Cauchy-Schwarz-Ungleichung}
 \item $2|x\cdot y| \leq \epsilon x^2 + \frac{1}{\epsilon}y^2, \epsilon > 0$ -- \emph{Young-Ungleichung}
\end{enumerate}

\subsubsection{Exponentialfunktion und Potenzen}
\paragraph{Exponentialfunktion} Im Folgenden gilt $x \in\mathbb{R}$.
\begin{enumerate}[label={(}\arabic*{)}]
 \item $e^x := Exp(x)$, definiert über Reihe, siehe unten.
 \item $e^x > 0, \forall x\in\mathbb{R}$
 \item $e^{-x} = \frac{1}{e^x}$
 \item $e^0 = 1, e^1 = e \approx 2.718281$
 \item $e^{-\infty} = 0, e^{\infty} = \infty$
 \item $e^{ix} = \cos(x) + i\sin(x)$ – \emph{Eulerformel}
 \item $e^{i\pi} = -1$ – \emph{Euleridentität}
 \item $e^{-1}(x) = \ln(x)$ also $e^{\ln(x)} = x = \ln(e^x)$.
\end{enumerate}
\paragraph{Potenzen} Im Folgenden gilt $a,b,n,m \in\mathbb{R}$.
\begin{enumerate}[label={(}\arabic*{)}]
 \item $a^x = e^{ln(a)^{x}} = e^{ln(a)x}$
 \item $a^{n+m} = a^na^m$
 \item $a^{nm} = (a^n)^m \neq a^{(n^m)}$
 \item $(ab)^n = a^nb^n$
 \item $(\frac{a}{b})^n = \frac{a^n}{b^n}$
\end{enumerate}
\paragraph{Wurzeln} Im Folgenden gilt $a,b,n,m \in\mathbb{R}$.
\begin{enumerate}[label={(}\arabic*{)}]
 \item $\sqrt[n]{a} := a^{\frac{1}{n}}$
 \item $\sqrt[n]{ab} = \sqrt[n]{a}\sqrt[n]{b}$
 \item $\sqrt[m]{\sqrt[n]{a}} = \sqrt[nm]{a}$
 \item $\sqrt[n]{\frac{a}{b}} = \frac{\sqrt[n]{a}}{\sqrt[n]{b}}$
\end{enumerate}

\subsubsection{Logarithmen} Im Folgenden gilt $a,r,x,y \in\mathbb{R}$.
\begin{enumerate}[label={(}\arabic*{)}]
 \item $\ln(x) := Exp^{-1}(x)$, also $x > 0$.
 \item $\ln(1) = 0, \ln(e) = 1$
 \item $\log_a(x) := \frac{\ln(x)}{\ln(a)}$
 \item $\log_a(\infty) = \infty$
 \item $\log_a(xy) = \log_a(x)+\log_a(y)$
 \item $\log_a(\frac{1}{x}) = -\log_a(x)$
 \item $\log_a(x^r) = n\log_a(x)$
 \item $\log_a(x \pm y) = \log_a(x) + \log_a(1 \pm \frac{y}{x})$
\end{enumerate}

\subsubsection{Komplexe Zahlen $\mathbb{C}$}
Sei $a,b \in \mathbb{R}, c \in \mathbb{C}$.
\begin{enumerate}[label={(}\arabic*{)}]
 \item $c := a + ib = \Re(a) + i\Im(b)$
 \item $\overline{c} = a - ib$ -- \emph{konjugiert-komplexe Zahl}
 \item $z_0 + z_1 := (a_0 + a_1) + i(b_0 + b_1)$
 \item $z_0 \cdot z_1 := (a_0a_1 - b_0b_1) + i(a_0b_1 + a_1b_0)$
 \item $|z|^2 = z\cdot\overline{z} = a^2 + b^2$
\end{enumerate}


\subsection{Trigonometrische Funktionen}
\paragraph{Wichtige Werte}
\begin{center}
\begin{tabular}[m]{r|ccccccccc}
Winkel & $\frac{1}{6}\pi$ & $\frac{1}{4}\pi$ & $\frac{1}{3}\pi$ & $\frac{1}{2}\pi$ & $\frac{2}{3}\pi$ & $\frac{3}{4}\pi$ & $\pi$ & $\frac{3}{2}\pi$ & $2\pi$ \\[.1em]
in Grad & $30$ & $45$ & $60$ & $90$ & $120$ & $135$ & $180$ & $270$ & $360$ \\\hline
$\sin(x)$ & $\frac{\sqrt{1}}{2}$ & $\frac{\sqrt{2}}{2}$ & $\frac{\sqrt{3}}{2}$ & $1$ & $\frac{\sqrt{3}}{2}$ & $\frac{\sqrt{2}}{2}$ & $0$ & $-1$ & $0$ \\[.1em]
$\cos(x)$ & $\frac{\sqrt{3}}{2}$ & $\frac{\sqrt{2}}{2}$ & $\frac{\sqrt{1}}{2}$ & $0$ & $-\frac{\sqrt{1}}{2}$ & $-\frac{\sqrt{2}}{2}$ & $-1$ & $0$ & $1$ \\[.1em]
$\tan(x)$ & $\frac{1}{\sqrt{3}}$ & $1$ & $\sqrt{3}$ & $0$ & $-\sqrt{3}$ & $-1$ & $0$ & $\times$ & $0$ \\
\end{tabular}
\end{center}

\paragraph{Rechenregeln}
\begin{enumerate}[label={(}\arabic*{)}]
 \item $\sin(x) := \sum_0^\infty (-1)^n\frac{x^{2n+1}}{(2n+1)!}$
 \item $\cos(x) := \sum_0^\infty (-1)^n\frac{x^{2n}}{(2n)!}$
 \item $\tan(x) := \frac{\sin(x)}{\cos(x)}$
 \item $\cos(x) + i\sin(x) = e^{ix}$
 \item $\sin(x)^2 + \cos(x)^2 = 1$
 \item $\sin(x \pm y) = \sin(x)\cos(y) \pm \cos(x)\sin(y)$
 \item $\cos(x \pm y) = \cos(x)\cos(y) \mp \sin(x)\sin(y)$
 \item $\tan(x \pm y) = \frac{\tan(x)\pm\tan(y)}{1\mp\tan(x)\tan(y)}$
 \item $\sin(2x) = 2\sin(x)\cos(x)$
 \item $\cos(2x) = 2\sin(x)\cos(x)$
 \item $\tan(2x) = \cos(x)^2 - \sin(x)^2 = 2\cos(x)^2 - 1 = 1 - 2\sin(x)^2$
 \item $\sin(x \pm \frac{\tau}{4}) = \sin(x\pm\frac{\pi}{2}) = \pm\cos(x)$
 \item $\cos(x \pm \frac{\tau}{4}) = \cos(x\pm\frac{\pi}{2}) = \mp\sin(x)$
 \item $\sin(x \pm \frac{\tau}{2}) = \sin(x\pm\pi) = -\sin(x)$
 \item $\cos(x \pm \frac{\tau}{2}) = \cos(x\pm\pi) = -\cos(x)$
\end{enumerate}

\subsection{Hyperbelfunktionen}
\begin{enumerate}[label={(}\arabic*{)}]
 \item $\sinh(x) := \frac{1}{2}(e^x-e^{-x}) = -i\sin(ix)$
 \item $\cosh(x) := \frac{1}{2}(e^x+e^{-x}) = \cos(ix)$
 \item $\tan(x) := \frac{\sin(x)}{\cos(x)} = 1 - \frac{2}{e^{2x}+1}$
 \item $\arcsinh(x) := \sinh^{-1}(x) = \ln(x + \sqrt{x^2+1})$
 \item $\arccosh(x) := \cosh^{-1}(x) = \ln(x + \sqrt{x^2-1})$
 \item $\arctanh(x) := \tanh^{-1}(x) = \frac{1}{2}\ln(\frac{1+x}{1-x})$
\end{enumerate}

\subsection{Folgen mit Grenzwerten}
Folgende Folgen sind sortiert nach ``Wachstumsschnelligkeit''.
\[ (1), (\ln(n)), (n^{a}), (q^n), (n!), (n^n) \text{ mit } a > 0, q > 1.\]
Im Folgenden ist $a_n \rightarrow a$ gleichbedeutend mit $\lim_{n\rightarrow\infty} a_n = a$.
Außerdem seien $a,k\in\mathbb{R}$ Konstanten.
\paragraph{Konvergente Folgen}
\begin{enumerate}[label={(}\arabic*{)}]
 \item $\sqrt[n]{a} \rightarrow 1, \sqrt[n]{n} \rightarrow 1, a \geq 0$
 \item $\frac{n}{\sqrt[n]{n!}} \rightarrow e, \frac{\sqrt[n]{n!}}{n} \rightarrow \frac{1}{e}$
 \item $(\frac{n+1}{n})^n \rightarrow e$
 \item $(1 + \frac{a}{n})^n \rightarrow e^a$
 \item $(a^nn^k)^n \rightarrow 0, |a| < 1$
 \item $n(\sqrt[n]{a}-1) \rightarrow ln(a), a > 0$
\end{enumerate}
\paragraph{Divergente Folgen}
\[ (\sqrt[n]{n!}), (\frac{n^n}{n!}), (\frac{a^n}{n^k}) \]
\paragraph{Bernoullische Ungleichung}
\[ \forall x\geq -1, n\in\N: (1+x)^n \geq 1 + nx \]

\subsection{Reihen mit Grenzwerten}
Sei mal $\sum_{n_0}$ Abkürzung für $\sum_{n=n_0}^\infty$.
\begin{enumerate}[label={(}\arabic*{)}]
 \item $\sum_1 \frac{1}{n}$ divergiert -- \emph{harmonische Reihe}
 \item $\sum_1 (-1)^n\frac{1}{n} = \ln(\frac{1}{2})$ -- \emph{alternierende harmonische Reihe}
 \item $\sum_1 \frac{1}{n^a}$ konvergiert für $a > 1$, sonst divergent.
 \item $\sum_0 q^n = \frac{1}{1-q}, |q| < 1$ -- \emph{geometrische Reihe}
 \item $\sum_0 q^n = \frac{1}{1+q}, |q| < 1$ -- \emph{alternierende geometrische Reihe}
 \item $\sum_1 \frac{1}{n^2} = \frac{\pi^2}{6}$
\end{enumerate}
\paragraph{Partialsummen}
\begin{enumerate}[label={(}\arabic*{)}]
 \item $\sum_{i=0}^n i = \frac{n(n-1)}{2}$ -- \emph{kleiner Gauß}
 \item $\sum_{i=0}^n i^2 = \frac{1}{6}n(n+1)(2n+1)$
 \item $\sum_{i=0}^n i^3 = \frac{1}{4}n^2(n+1)^2$
 \item $\sum_{i=0}^n q^n = \frac{1-q^{n+1}}{1-q}$
\end{enumerate}

\clearpage

\subsection{Ableitungen}
Im Folgenden sei $f(x) \rightarrow g(x)$ Abkürzung für $\frac{d}{dx} f(x) = g(x)$.
\subsubsection{Rechenregeln}
\begin{enumerate}[label={(}\arabic*{)}]
 \item $(f(x)+g(x))' = f'(x) + g'(x)$
 \item $(f(x)g(x))' = f'(x)g(x) + f(x)g'(x)$
 \item $(\frac{z(x)}{n(x)})' = \frac{z(x)n'(x) - z'(x)n(x)}{n(x)^2}$
 \item $(g \circ f)'(x) = (g(f(x)))' = f'(x)g'(f(x))$
\end{enumerate}
\subsubsection{Polynome und Wurzeln}
\begin{enumerate}[label={(}\arabic*{)}]
 \item $x^a \rightarrow ax^{a-1}$
 \item $\frac{1}{x^a} = x^{-a} \rightarrow -ax^{-a-1} = \frac{-a}{x^{a+1}}$
 \item $\sqrt[a]{x^b} = x^{\frac{b}{a}} \rightarrow \frac{b}{a}x^{\frac{b}{a}-1}$
\end{enumerate}
\subsubsection{Exponenten und Logarithmen}
\begin{enumerate}[label={(}\arabic*{)}]
 \item $e^{ax} \rightarrow ae^{ax}$
 \item $e^{x^{a}} \rightarrow ax^{a-1}e^{x^{a}}$
 \item $a^{x} = e^{ln(a)^{x}} = e^{\ln(a)x} \rightarrow \ln(a)\cdot a^x$
 \item $x^{x} \rightarrow (1+\ln(x))x^{x}$
 \item $x^{x^a} \rightarrow (1+a\ln(x))x^{x^a+a-1}$
 \item $\ln(x) \rightarrow \frac{1}{x}$
 \item $\log_a(x) = \frac{1}{\ln(a)}\ln(x) \rightarrow \frac{1}{\ln(a)x}$
\end{enumerate}
\subsubsection{Trigonometrische Funktionen}
\begin{enumerate}[label={(}\arabic*{)}]
 \item $\sin(x) \rightarrow \cos(x)$
 \item $\cos(x) \rightarrow -\sin(x)$
 \item $\sin(ax+b) \rightarrow a\cos(ax+b)$
 \item $\tan(x) \rightarrow \frac{1}{(\cos(x))^2}$
 \item $\arcsin(x) \rightarrow \frac{1}{\sqrt{1-x^2}}$
 \item $\arccos(x) \rightarrow \frac{-1}{\sqrt{1-x^2}}$
 \item $\arctan(x) \rightarrow \frac{1}{x^2+1}$
 \item $\sinh(x) \rightarrow \cosh(x)$
 \item $\cosh(x) \rightarrow \sinh(x) \neq -\sinh(x)$!
 \item $\tanh(x) \rightarrow \frac{1}{(\cosh(x))^2}$
 \item $\arcsinh(x) \rightarrow \frac{1}{\sqrt{x^2+1}}$
 \item $\arccosh(x) \rightarrow \frac{1}{\sqrt{x-1}\sqrt{x+1}}$
 \item $\arctanh(x) \rightarrow \frac{1}{1-x^2}$
\end{enumerate}

\subsection{Unbestimmte Integrale}
\subsubsection{Rechenregeln}
\begin{enumerate}[label={(}\arabic*{)}]
 \item $\int(f(x)+g(x))dx = \int f(x)dx + \int g(x)dx$
 \item $\int af(x)dx = a\int f(x)dx$
 \item $\int u'(x)v(x)dx = u(x)v(x) - \int u(x)v'(x)dx$
 \item $\int f(g(x))g'(x)dx = \int f(x)dx$
 \item $\int f(ax+b)dx = \frac{1}{a}F(x+b)$
 \item $\int \frac{f'(x)}{f(x)}dx = \ln(|f(x)|)$
 \item $\int f'(x)f(x)dx = \frac{1}{2}f(x)^2$
 \item $\int |f(x)|dx = |\int f(x)dx|$
\end{enumerate}
\subsubsection{Polynome und Wurzeln}
\begin{enumerate}[label={(}\arabic*{)}]
 \item $\int x^a dx = \frac{x^{a+1}}{a+1}$
 \item $\int \frac{1}{x^a} dx = \int x^{-a} dx = \frac{x^{-a+1}}{-a+1} = -\frac{a-1}{x^{a-1}}, a \neq 1$
 \item $\int \sqrt[a]{x^b} dx = \int x^{\frac{b}{a}} dx \rightarrow \frac{x^{\frac{b}{a}+1}}{\frac{b}{a}+1} = \frac{a}{b+a}\sqrt[a]{x^{b+a}}$
\end{enumerate}
\subsubsection{Exponenten und Logarithmen}
\begin{enumerate}[label={(}\arabic*{)}]
 \item $\int e^{ax} dx = \frac{1}{a}e^{ax}$
 \item $\int xe^{x} dx = (x-1)e^{x}$
 \item $\int a^{x} dx = \int e^{ln(a)x} dx = \frac{1}{ln(a)}a^x$
 \item $\int \frac{1}{x} dx = \ln(x)$
 \item $\int \ln(x) dx = x(\ln(x)-1)$
\end{enumerate}
\subsubsection{Trigonometrische Funktionen}
\begin{enumerate}[label={(}\arabic*{)}]
 \item $\int \sin(x) dx = -\cos(x)$
 \item $\int \cos(x) dx = \sin(x)$
 \item $\int \sin(ax+b) dx = -\frac{1}{a}\cos(ax+b)$
 \item $\int \tan(x) dx = -\ln(|\cos(x)|)$
 \item $\int \arcsin(x) dx = x\arcsin(x) + \sqrt{1-x^2}$
 \item $\int \arccos(x) dx = x\arccos(x) - \sqrt{1-x^2}$
 \item $\int \arctan(x) dx = x\arctan(x) - \frac{1}{2}\ln{1+x^2}$
 \item $\int \sinh(x) dx = \cosh(x)$
 \item $\int \cosh(x) dx = \sinh(x) \neq -\sinh(x)$
 \item $\int \tanh(x) dx = \ln(\cosh(x))$
 \item $\int \arcsinh(x) dx = x\arcsinh(x) + \sqrt{x^2+1}$
 \item $\int \arccosh(x) dx = x\arccosh(x) + \sqrt{x-1}\sqrt{x+1}$
 \item $\int \arctanh(x) dx = x\arctanh(x) + \frac{1}{2}\ln(1-x^2)$
\end{enumerate}

\clearpage
\subsection{Hilfen für Diff'rechnung in $\mathbb{R}^n$}
\subsubsection{Koordinatentransformationen}
\paragraph{Kugelkoordinaten in $\R^3$}
\[
\begin{pmatrix}
 x \\ y \\ z \\
\end{pmatrix}
=
\begin{pmatrix}
 r\sin(\theta)\cos(\varphi) \\
 r\sin(\theta)\sin(\varphi) \\
 r\cos(\theta)
\end{pmatrix}
\]
\[
\begin{pmatrix}
  r \\ \theta \\ \varphi \\
\end{pmatrix}
=
\begin{pmatrix}
  \sqrt{x^2 + y^2 + z^2} \\
  \arccos(\frac{z}{r}) \\
  \operatorname{atan2} (y, x) \\
\end{pmatrix}
\]
Mit $\operatorname{atan2} (y, x)= 2 \arctan \dfrac{y}{\sqrt{x^2+y^2}+x} = \arctan(\dfrac{y}{x}) \\$\\
Jacobi-Matrix:
\[
J =
\begin{pmatrix}
 \sin(\theta)\cos(\varphi) & r\cos(\theta)\cos(\varphi) & -r\sin(\theta)\sin(\varphi) \\
 \sin(\theta)\sin(\varphi) & r\cos(\theta)\sin(\varphi) & r\sin(\theta)\cos(\varphi) \\
 \cos(\theta) & r\sin(\theta) & 0 \\
\end{pmatrix}
\]
Jacobi-Determinante: $\det(J) = r^2\sin(\theta)$\\
Volumenelement: $dV = r^2\sin(\theta)\, d\varphi\, d\theta\, dr$

\paragraph{Zylinderkoordinaten in $\R^3$}
\[
\begin{pmatrix}
 x \\ y \\ z \\
\end{pmatrix}
=
\begin{pmatrix}
 r\cos(\varphi) \\
 r\sin(\varphi) \\
 z
\end{pmatrix}
\]
Jacobi-Matrix:
\[
J =
\begin{pmatrix}
 \cos(\varphi) & -r\sin(\varphi) & 0 \\
 \sin(\varphi) &  r\cos(\varphi) & 0 \\
 0 & 0 & 1 \\
\end{pmatrix}
\]
Jacobi-Determinante: $\det(J) = r\cos(\varphi)^2+r\sin(\varphi)^2 = r$\\
Volumenelement: $dV = r\, dr\, d\varphi\, dz$

\subsubsection{Typische geometrische Körper und ihre Volumina}
\paragraph{Ellipsoid}
Ein \emph{Ellipsoid} ist -- in kartesischen Koordinaten im $\R^3$ -- gegeben durch
\[ \frac{x^2}{a^2} + \frac{y^2}{b^2} + \frac{z^2}{c^2} = 1. \]
$a,b,c$ nennt man dabei die \emph{Halbachsen} der Ellipse.

\paragraph{Volumina und Oberflächen}

\begin{tabular}{lll}
 \textbf{Körper} & $V$ \\\hline
 Kugel & $\frac{4\pi}{3}r^3$ \\
 Ellipsoid & $\frac{4\pi}{3}abc$ \\
 Zylinder & $\pi r^2h$ \\
 Pyramide & $\frac{1}{3}Gh$ \\
 Kegel & $\frac{\pi}{3}r^2h$ \\
\end{tabular}

\end{document}
